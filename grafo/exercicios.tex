\documentclass[12pt]{article}
\usepackage{fontspec}

\usepackage{enumerate}
\usepackage{tikz}
\usepackage{verbatim}
\usetikzlibrary{arrows,automata,shapes}

\begin{document}

\title{Caminhos em grafos}
\author{Adriano J. Holanda}
\date{\today}

\paragraph{Exercício 1.}
Suponha que o algoritmo de Dijkstra seja executado sobre o seguinte
grafo, começando pelo vértice {\tt a} e terminando pelo vértice {\tt
  h}. (Adaptado de~\cite{dasgupta2009}, pág. 120) \bigskip

\tikzstyle{vertex}=[circle,fill=black!25,minimum size=20pt,inner sep=0pt]
\tikzstyle{selected vertex} = [vertex, fill=blue!24]
\tikzstyle{edge} = [draw,thick,->]
\tikzstyle{weight} = [font=\small]
\tikzstyle{selected edge} = [draw,line width=5pt,-,red!50]
\tikzstyle{ignored edge} = [draw,line width=5pt,-,black!20]

\begin{tikzpicture}[scale=1.8, auto,swap]
  \centering
    % Draw a 7,11 network
    % First we draw the vertices
    \foreach \pos/\name in {{(-3,0)/a},{(-1,0)/b},{(1,0)/c},{(3,0)/d},
                            {(-3,-2)/e}, {(-1,-2)/f}, {(1,-2)/g}, {(3,-2)/h}}
        \node[vertex] (\name) at \pos {$\name$};
    % Connect vertices with edges and draw weights
        \foreach \source/ \dest /\weight in {
          a/b/1, a/e/4,a/f/8,
          b/c/2,b/f/6,b/g/6,
          c/d/1,c/g/2,
          d/g/1,d/h/4,
          e/f/5,
          g/f/1,g/h/1}
        \path[edge] (\source) -- node[weight] {$\weight$} (\dest);
    % Start animating the vertex and edge selection.
    \foreach \vertex / \fr in {d/1,a/2,f/3,b/4,e/5,c/6,g/7}
    \path node[selected vertex] at (\vertex) {$\vertex$};
  \label{fig:dijk0}
\end{tikzpicture}

\begin{enumerate}
\item Desenhe uma tabela mostrando os valores intermediários de
  distância para rodos os vértices em cada iteração do algoritmo. Faça
  a tabela usando a implementação do algoritmo com matriz de
  adjacências e lista de adjacências.
\item Mostra a árvore final de caminho mínimo.
\end{enumerate}

\paragraph{Exercício 2.} Utilizando o grafo a seguir:~\cite{boaventura2009}

\begin{enumerate}[a)]
\item Aplique o algoritmo de Dijkstra para achar a menor distância do vértice
  {\tt a} aos outros vértices.
\item Construa a arborescência de distância a partir de {\tt a}.
\end{enumerate}\bigskip

\begin{tikzpicture}[scale=1.8, auto,swap]
  \centering
    % Draw a 7,11 network
    % First we draw the vertices
  \foreach \pos/\name in {{(-3,0)/a},{(-1,1)/b},{(-1,0)/c},{(-1,-1)/d},
  {(1,1)/e},{(1,0)/f},{(1,-1)/g},{(1.5,-2)/h},{(2.5,0)/i},{(3.5,0)/j}}
        \node[vertex] (\name) at \pos {$\name$};
    % Connect vertices with edges and draw weights
        \foreach \source/ \dest /\weight in {
          a/b/4,a/c/1,a/d/3,
          b/c/1,b/e/3,
          c/d/1,c/e/7,c/f/6,
          d/f/2,d/g/3,d/h/2,
          e/f/1,e/i/2,
          f/i/4,
          g/f/1,g/h/1,g/i/4,
          h/j/2,
          i/j/1}
        \path[edge] (\source) -- node[weight] {$\weight$} (\dest);
    % Start animating the vertex and edge selection.
    \foreach \vertex / \fr in {d/1,a/2,f/3,b/4,e/5,c/6,g/7}
    \path node[selected vertex] at (\vertex) {$\vertex$};
  \label{fig:dijk0}
\end{tikzpicture}

\paragraph{Exercício 3.}

\begin{center}
  \begin{tabular}[hf]{ccccccccc}
     & 1 & 2  & 3  & 4  & 5  & 6  & 7  & 8  \\
    1 & 0 & 18 & 45 & 27 & 81 & 54 & 36 & 90 \\
    2 & 42 & 0 & 21 & 49 & 35 & 28 & 14 & 56 \\
     3 & 15 & 35 & 0 & 50 & 20 & 25 & 55 & 30 \\
    4 & 32 & 64 & 40 & 0 & 16 & 56 & 24 & 48 \\
     5 & 30 & 42 & 54 & 18 & 0 & 24 & 48 & 36 \\
     6 & 24 & 44 & 36 & 48 & 32 & 0 & 56 & 28 \\
     7 & 21 & 36 & 48 & 27 & 42 & 57 & 0 & 24 \\
     8 & 98 & 48 & 88 & 72 & 40 & 80 & 64 & 0 \\
  \end{tabular}
\end{center}

\begin{thebibliography}{9}
\bibitem{dasgupta2009}
    {Sanjoy Dasgupta, Christos Papadimitriou, Umesh Vazirani}.
    \emph{Algoritmos}.
    McGraw-Hill,
    2009.

\bibitem{boaventura2009}
  {Paulo Oswaldo Boaventura Netto, Samuel Jurkiewicz}.
  \emph{Grafos: Introdução e Prática}.
  Editora Blucher, 1ª edição, 2009.

\end{thebibliography}

\end{document}