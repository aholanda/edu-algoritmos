\input cs-schola

\font\tenib=cmmib10
\font\sevenib=cmmib7
\font\fiveib=cmmib5
\font\tensyb=cmbsy10
\font\sevensyb=cmbsy7
\font\fivesyb=cmbsy5

\def\b#1{\bf\tenib \underline{#1}}

{\bf Inversões}

Seja $a_1, a_2, \ldots, a_n$ uma permutação do conjunto
$\{1,2,\ldots,n\}$. Se $i<j$ e $a_i>a_j$, o par $(a_i, a_j)$ é chamado
{\it inversão} da permutação; por exemplo, a permutação 3 1 4 2 tem
três inversões: $(3,1)$, $(3,2)$ e $(4,2)$.

A {\it tabela de inversão} $b_1, b_2, \ldots, b_n$ é obtida deixando
$b_j$ ser o número de elementos a esquerda de $j$ que são maiores do
que $j$.

Por exemplo, a permutação

$$5\quad 9\quad 1\quad 8\quad 2\quad 6\quad 4\quad 7\quad 3\eqno(1)$$

\noindent tem a tabela de inversão

$$2\quad 3\quad 6\quad 4\quad 0\quad 2\quad 2\quad 1\quad 0,$$

\noindent pois 5 e 9 estão a direita de 1; 5, 9, 8 estão a direita de
2 e assim sucessivamente.

O número $b_j$ para que este tenha o valor máximo é $n-j$, a permutação
para $b_j$ máximo, sendo $1\leq j\leq n$ para a permutação $(1)$ seria

$$9\quad 8\quad 7\quad 6\quad 5\quad 4\quad 3\quad 2\quad 1$$

\noindent e a tabela de inversão com os valores máximos tendo $n=9$

$$b_j\qquad 8\quad 7\quad 6\quad 5\quad 4\quad 3\quad 2\quad 1\quad 0$$
$$j\qquad 1\quad 2\quad 3\quad 4\quad 5\quad 6\quad 7\quad 8\quad 9$$


Portanto a permutação $(1)$ possui $b_j$ máximo em $j=\{3, 7, 8,
9\}$. Sendo assim, $b_1$ será igual a $n-1$ com probabilidade $1/n$,
$b_2$ igual a $n-2$ com probabilidade $1/(n-1)$ e assim sucessivamente,
onde obtemos o número médio de máximos por

$${1 \over n} + {1\over n-1}+\ldots 1=H_n.$$

\noindent Se tomarmos como exemplo o conjunto $\{1,2,3\}$ com $n=3$, obtemos o
número médio de máximos

$$\mu = H_3 = {1\over3} + {1\over2} + 1={2+3+6\over 6}=11/6,$$

\noindent e os máximos $b_j$ são sublinhados nas permutações de
$\{1,2,3\}$ a seguir

$$1\ 2\ 3 \rightarrow 0\quad 0\quad \b{0}$$
$$1\ 3\ 2 \rightarrow 0\quad \b{1}\quad \b{0}$$
$$2\ 1\ 3 \rightarrow 1\quad 0\quad \b{0}$$
$$2\ 3\ 1 \rightarrow \b{2}\quad 0\quad \b{0}$$
$$3\ 1\ 2 \rightarrow 1\quad \b{1}\quad \b{0}$$
$$3\ 2\ 1 \rightarrow \b{2}\quad \b{1}\quad \b{0}$$

\noindent onde obtemos seis $b_3=0$, três $b_2=1$ e dois $b_1=2$ totalizando
$11$ máximos em $6$ permutações possíveis, ou seja, $11/6$.

\vfill

\noindent Referência: Donald Knuth, {\it The Art of Computer
  Programming, vol.~3}, 2nd edition. Addison Wesley, 1997.

\bye
