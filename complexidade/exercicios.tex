\documentclass{article}

\usepackage{fontspec}
\usepackage{listings}
\usepackage{enumerate}
\usepackage[]{polyglossia}
\setdefaultlanguage{brazil}

\title{Complexidade de Algoritmos: Notação $O$.}
\author{Adriano J. Holanda, Zhao Liang}
\date{\today}

\lstset{language=C}

\begin{document}
\maketitle
\paragraph{Exercício 1.} Ache a ordem de complexidade dos fragmentos a
seguir:

\begin{enumerate}[a)]

\begin{minipage}{.6\textwidth}
\item
  \begin{lstlisting}
  int i, soma = 0;
  for (i=0; i<n; i++)
      soma++;
  \end{lstlisting}
  $O(n)$
\end{minipage}
\begin{minipage}{.4\textwidth}
\item
\begin{lstlisting}
  int i, soma = 0;
  for (i=n; i>1; i /= 2)
      soma++;
\end{lstlisting}
  $O(\lg n)$
\end{minipage}
\begin{minipage}{.6\textwidth}
\item
\begin{lstlisting}
  int i, j, soma = 0;
  for (i=0; i<n; i++)
    for (j=0; j<n; j++)
          soma++;
\end{lstlisting}
  $O(n^2)$
\end{minipage}
\begin{minipage}{.4\textwidth}
  \item
\begin{lstlisting}
  int i, j, soma = 0;
  for (i=n; i>1; i /= 2)
    for (j=0; j<n; j++)
          soma++;
\end{lstlisting}
  $O(n\lg n)$
\end{minipage}
\begin{minipage}{.6\textwidth}
\item
\begin{lstlisting}
  int i, j, soma = 0;
  for (i=n; i>1; i /= 2)
    for (j=i; j<n; j++)
          soma++;
\end{lstlisting}
  $O(n^2)$
\end{minipage}
\begin{minipage}{.4\textwidth}
\item
\begin{lstlisting}
  int i, j, soma = 0;
  for (i=n; i>1; i /= 2)
    for (j=0; j<i; j++)
          soma++;
\end{lstlisting}
  $O(n\lg n)$
\end{minipage}

\end{enumerate}

\noindent Lembrando que $\lg n = \log_2(n)$.

\paragraph{Exercício 2.} Ache a ordem de complexidade das expressões a seguir usando a notação-O:

\begin{enumerate}[a)]
\item $5 + 0,001n^3 + 0.25n=O(n^3)$
\item $500n + 300n^{1.5}+30\log_{10}n=O(n^{1,5})$
\item $0,3n+5n^{1,5}+2,5n^{1,75}=O(n^{1,75})$
\item $n\log_2n + n\log_3n=O(n\log_2n)$
\item $5000n + 0,005n^2=O(n^2)$
\item $3\log_2n + \log_2\log_2\log_2n=O(\log_2n)$
\end{enumerate}

\end{document}
