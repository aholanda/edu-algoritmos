\documentclass{article}


\usepackage{algorithmic}
\usepackage{enumerate}
\usepackage{fontspec}
\usepackage{polyglossia}
\setmainlanguage{brazil}

\begin{document}

\title{Laboratório: Complexidade de Algoritmos - Recorrências}
\author{Adriano J. Holanda e Zhao Liang}
\maketitle

\paragraph{1.}~Implemente uma função recursiva em C que dado um vetor de inteiros {\tt v[N]},
 ache o índice do maior elemento.

\paragraph{2.}~Implemente uma função recursiva em C que dado um vetor de inteiros {\tt v[N]},
 busque o índice de um elemento {\tt x} no vetor {\tt v}.

 \paragraph{3.}~Para cada uma das funções recorrentes a seguir, faça uma
 função em C que as implemente:

\begin{enumerate}[a)]
\item Soma harmônica $\Rightarrow$ $1+1/2+1/3+\ldots+1/N$:\\
    $$\cases{H(1)=1,\cr H(N) = H(N-1) + 1/N .}$$
\item Soma triangular $\Rightarrow$ $1 + 2 + 3 + \ldots + N$:\\
  $$\cases{T(1)=1,\cr T(N) = T(N-1) + N .\cr}$$
\item Aproximação de Stirling $\Rightarrow$ $lg1 + lg2 + lg3+\ldots+lg\,N$:\\
    $$\cases{S(1)=lg\,1, \cr S(N) = S(N-1) + lg\,N .\cr}$$
  Dica: o $log_2$ pode ser obtido usando a função logaritmo natural {\tt log(x)} da
  biblioteca matemática de C da seguinte forma:
  $$log_2(x)=log(x)/log(2).$$
\end{enumerate}

\end{document}

\title{Laboratório: Complexidade de Algoritmos}

\paragraph{1.}~Implemente usando a linguagem C, os algoritmos
descritos em pseudocódigos. Para cada programa calcule o tempo de
execução do algoritmo no programa usando a funcão {\tt clock()}
da biblioteca {\tt time.h} para os seguintes valores de $N$:

$$\{32,64,128,256,1024,2056\}.$$

\begin{itemize}
\item Faça uma tabela usando o programa {\tt libreoffice-calc}
  contendo o tamanho da entrada de dados e o tempo de execução do
  algoritmo.
\item Faça um gráfico de dispersão XY acessando \hbox{\tt
  Inserir->Gráfico...->XY (Dispersão)} e escolhendo o segundo gráfico
  disponível.
\item Faça uma tentativa de ajuste da curva para o algoritmo da soma
  selecionando as colunas {\tt N} e de tempo da soma e acessando \\
  \hbox{\tt Dados->Estatística->Regressão...->Regressão Linear}.
\item Para os outros algoritmos, selecione os dados de tempo do
  algoritmo e acesse \hbox{\tt
    Dados->Estatística->Regressão...->Regressão Geométrica}, tomando
  o cuidado para remover a seleção da {\tt Regressão Linear}.
  Na {\it Regressão Geométrica\/}, o expoente a ser analisada deve
  ser comparado com a inclinação obtida para a reta.
  \item Verifique os valores gerados para as curvas estão próximos
    da análise assintótica.
\end{itemize}

\begin{enumerate}[a)]
\item Algoritmo para o cálculo do somatório $\sum_{i=0}^N i^3$:

  \begin{algorithmic}
    \STATE $soma\gets 0$
    \STATE $i\gets 0$
    \REPEAT
    \STATE $soma \gets soma + i^3$
    \UNTIL{$i\leq N$}
  \end{algorithmic}
\pagebreak
\item Algoritmo para a soma de 2 matrizes $A_{n,n}$ e $B_{n,n}$ resultando em uma matrix $C_{n,n}$:

  \begin{algorithmic}
    \STATE $i\gets 0$
    \STATE $j\gets 0$
    \WHILE{$i<N$}
    \WHILE{$j<N$}
    \STATE $a_{i,j}\gets i+j$
    \COMMENT{{\small Atribui a soma dos indices para elemento da matriz.}}
    \STATE $b_{i,j}\gets i+j$
    \COMMENT{{\small Atribui a soma dos indices para elemento da matriz.}}
    \STATE $c_{i,j}\gets a_{i,j} + b_{i,j}$
    \ENDWHILE
    \STATE{$j\gets 0$}
    \STATE{$i\gets i + 1$}
    \ENDWHILE
  \end{algorithmic}

\item Algoritmo para a multiplicação de 2 matrizes $A_{n,n}$ e
  $B_{n,n}$ resultando em uma matrix $C_{n,n}$:

  \begin{algorithmic}
    \STATE $i\gets 0$
    \STATE $j\gets 0$
    \STATE $k\gets 0$
    \WHILE{$i<N$}
    \WHILE{$j<N$}
    \STATE{$c_{i,j}\gets 0$}
    \WHILE{$k<N$}
    \STATE $a_{i,k}\gets i+k$
    \COMMENT{{\small Atribui a soma dos indices para elemento da matriz.}}
    \STATE $b_{k,j}\gets k+j$
    \COMMENT{{\small Atribui a soma dos indices para elemento da matriz.}}
    \STATE $c_{i,j}\gets c_{i,j} + a_{i,k} * b_{k,j}$
    \ENDWHILE
    \STATE{$k\gets 0$}
    \STATE{$j\gets j + 1$}
    \ENDWHILE
    \STATE{$j\gets 0$}
    \STATE{$i\gets i + 1$}
    \ENDWHILE
  \end{algorithmic}

\end{enumerate}

\paragraph{2.} Usando os mesmos valores de {\tt N} do exercício anterior, faça
uma tabela e gráfico de dispersão XY no {\tt libreoffice} para as
funções:

\begin{enumerate}[a)]
\item $f(n) = log_2n$
\item $f(n) = n$
\item $f(n) = n^2$
\item $f(n) = n^3$
\end{enumerate}

\end{document}
