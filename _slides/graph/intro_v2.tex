
\title{Grafos}

\def\GG{
	\begin{tikzpicture}[vertex/.style={circle,draw}]
	\node[vertex] (v0) at (0,0) {$0$};
	\node[vertex] (v1) at (-1,-1) {$1$};
	\node[vertex] (v2) at (1,-1) {$2$};
	\node[vertex] (v3) at (1.5,-2) {$3$};

	\graph {
       	       (v0) -- (v1) -- (v2) -- (v0);
	       (v2) -- (v3);
	 };

	 \node [above of=v0]{Grafo não direcionado};
	 \end{tikzpicture}
}

\def\GD {
	\begin{tikzpicture}[vertex/.style={circle,draw}]
	\node[vertex] (v0) at (0,0) {$0$};
	\node[vertex] (v1) at (-1,-1) {$1$};
	\node[vertex] (v2) at (1,-1) {$2$};
	\node[vertex] (v3) at (1.5,-2) {$3$};
	\path (v0) edge [loop above](v0); % self-loop

	\graph {
       	       (v0) -> (v1) -> (v2) -> (v0);
       	       (v2) -> (v3);
        };

	\node [above of=v0]{Dígrafo};
	\end{tikzpicture}
}

\frame{\maketitle}

\section{Conceitos básicos}

\begin{frame}{Grafos}{Definição}

\begin{definition} Um \alert{\only<2>{dí}grafo $G$} é formado por um
conjunto de \alert{vértices $V$} (ou nós, ou pontos) e uma coleção de
\alert{\only<1>{arestas}\only<2>{arcos} $E$} (\only<1>{ou {\it links\/}, ou
linhas}\only<2>{ou setas}), onde cada aresta conecta um par de vértices.
\end{definition}

\only<1>{
\begin{center}
\noindent Grafo \alert{não} direcionado\bigskip

\begin{tikzpicture}[font=\footnotesize]
\graph[nodes={draw,circle}] {
       1 -- 4 -- 5;
       0 -- {1, {2,3} -- 5};
       
};
\end{tikzpicture}\bigskip


\noindent $G = (V,E), \qquad V = \{0, 1, 2, 3, 4, 5\}$
\noindent $E = \{\{0,1\}, \{0,2\}, \{0,3\}, \{1,4\}, \{2,5\}, \{3,5\}, \{4,5\}\}$
\end{center}
}

\only<2>{
\begin{center}
\noindent Grafo direcionado (dígrafo)\bigskip

\begin{tikzpicture}[font=\footnotesize, new set=mynodes,
vertex/.style={circle,draw}]

\node[set=mynodes, vertex] (v0) at (0,0) {$0$};
\node[set=mynodes, vertex] (v1) at (-1,-1) {$1$};
\node[set=mynodes, vertex] (v2) at (0,-1) {$2$};
\node[set=mynodes, vertex] (v3) at (1,-.75) {$3$};
\node[set=mynodes, vertex] (v4) at (-.5,-2) {$4$};
\node[set=mynodes, vertex] (v5) at (0.5,-1.75) {$5$};

\graph {
       (v0) -> (v1) -> (v4) -> (v5);
       (v5) -> (v2) -> (v0);
       (v5) -> (v3) -> (v0);
};
\end{tikzpicture}


\bigskip

\noindent $G = (V,E), \qquad V = \{0, 1, 2, 3, 4, 5\}$
\noindent $E = \{\<0,1>, \<1,4>, \<4,5>, \<5,2>, \<5,3>, \<2,0>, \<3,0>\}$
\end{center}
}

\end{frame}

%% See Algorithms by Sedgewick
\begin{frame}{Aplicações}

\begin{itemize}
\item Mapas
\item Internet
\item Circuitos
\item Redes de computadores
\item Software
\item Redes sociais
\item $\ldots$
\end{itemize}

\end{frame}

\section{Representação}

%%%%%%%%%%%%%%%%%%%%%%%%%%%%%%%% MATRIX

\begin{frame}{Representação por matriz}{\inserttitle}

\begin{columns}

\begin{column}{.5\textwidth}

\GG

\end{column}

\begin{column}{.5\textwidth}

\noindent Matriz de adjascências\bigskip

\def\R{\color{red}}
\def\Z{{\color{gray}0}}
\begin{tabular}{ccccc}
		     &\R 0   &\R 1   &\R 2   & \R 3\\
\color{red}	   0 &  \Z   &   1   &   1   &   \Z \\
\color{red} 	   1 &   1   &  \Z   &   1   &   \Z \\
\color{red} 	   2 &   1   &   1   &  \Z   &    1 \\
\color{red} 	   3 &  \Z   &  \Z   &   1   &   \Z \\
\end{tabular}

\only<2> {
\vskip 1cm {\tt int mat[4][4];}
}
\end{column}
\end{columns}

\end{frame}


\section{Representação}

\begin{frame}{Representação por matriz}{\inserttitle}

\begin{columns}

\begin{column}{.5\textwidth}

\GD

\end{column}

\begin{column}{.5\textwidth}

\noindent Matriz de adjascências\bigskip

\def\R{\color{red}}
\def\Z{{\color{gray}0}}
\begin{tabular}{ccccc}
		     &\R 0   &\R 1   &\R 2  &\R 3\\
\color{red}	   0 & 	 1   &   1   &  \Z  &  \Z \\
\color{red} 	   1 &	\Z   &  \Z   &   1  &  \Z \\
\color{red} 	   2 & 	 1   &  \Z   &  \Z  &   1 \\
\color{red} 	   3 &  \Z   &  \Z   &  \Z  &  \Z \\
\end{tabular}

\only<2> {
\vskip 1cm {\tt int mat[4][4];}
}
\end{column}
\end{columns}

\end{frame}

%%%%%%%%%%%%%%%%%%%%%%%%%%%%%%%% LINKED LISTS

\def\NULL#1{
  \node[on chain,draw,inner sep=6pt] (N#1)[right of=#1,xshift=-.75cm]{};
  \draw (N#1.north east) -- (N#1.south west);
  \draw (N#1.north west) -- (N#1.south east);
  \draw[*->] let \p1 = (#1.two), \p2 = (#1.center) in (\x1,\y2) -- (N#1);
}

\def\EMPTYKEY#1{
  \node[on chain,draw,inner sep=6pt] (E#1)[right of=#1]{};
  \draw (E#1.north east) -- (E#1.south west);
  \draw (E#1.north west) -- (E#1.south east);
  \draw[*->] (#1) -- (E#1);
}

\begin{frame}{Representação por listas ligadas}

\begin{columns}

\begin{column}{.4\textwidth}

\GG

\end{column}

\begin{column}{.6\textwidth}
\begin{tikzpicture}[list/.style={rectangle split, rectangle split parts=2,
    draw, rectangle split horizontal}, >=stealth, start chain,
    every node/.style={xshift=.1cm}]

   \foreach \i in {0,1,2,3}{
    	     \node at (-.25,-\i) {$\i$};
	     \draw (0,-\i) +(-.5,-.5) rectangle ++(.5,.5);
	     \node (V\i) at (0,-\i) {};
  }


  \node[list,on chain] (E01) [right of=V0]{$1$};
  \node[list,on chain] (E02) [right of=E01,xshift=-.75cm]{$2$};
  \node[list,on chain] (E10) [right of=V1,xshift=-.75cm]{$0$};
  \node[list,on chain] (E12) [right of=E10,xshift=-.75cm]{$2$};
  \node[list,on chain] (E20) [right of=V2,xshift=-.75cm]{$0$};
  \node[list,on chain] (E21) [right of=E20,xshift=-.75cm]{$1$};
  \node[list,on chain] (E23) [right of=E21,xshift=-.75cm]{$3$};

  \draw[*->] (V0) -- (E01);
  \draw[*->] let \p1 = (E01.two), \p2 = (E01.center) in (\x1,\y2) -- (E02);
  \NULL{E02}

  \draw[*->] (V1) -- (E10);
 \draw[*->] let \p1 = (E10.two), \p2 = (E10.center) in (\x1,\y2) -- (E12);
  \NULL{E12}

  \draw[*->] (V2) -- (E20);
  \draw[*->] let \p1 = (E20.two), \p2 = (E20.center) in (\x1,\y2) -- (E21);
  \draw[*->] let \p1 = (E21.two), \p2 = (E21.center) in (\x1,\y2) -- (E23);
  \NULL{E23}

  \EMPTYKEY{V3}

\end{tikzpicture}
\end{column}
\end{columns}

\end{frame}


\begin{frame}{Representação por listas ligadas}

\begin{columns}

\begin{column}{.4\textwidth}
\GD
\end{column}

\begin{column}{.6\textwidth}
\begin{tikzpicture}[list/.style={rectangle split, rectangle split parts=2,
    draw, rectangle split horizontal}, >=stealth, start chain,
    every node/.style={xshift=.1cm}]

   \foreach \i in {0,1,2,3}{
    	     \node at (-.25,-\i) {$\i$};
	     \draw (0,-\i) +(-.5,-.5) rectangle ++(.5,.5);
	     \node (V\i) at (0,-\i) {};
  }


  \node[list,on chain] (E00) [right of=V0]{$0$};
  \node[list,on chain] (E01) [right of=E00,xshift=-.75cm]{$1$};
  \node[list,on chain] (E12) [right of=V1,xshift=-.75cm]{$2$};
  \node[list,on chain] (E20) [right of=V2,xshift=-.75cm]{$0$};
  \node[list,on chain] (E23) [right of=E20,xshift=-.75cm]{$3$};

  \draw[*->] (V0) -- (E00);
  \draw[*->] let \p1 = (E00.two), \p2 = (E00.center) in (\x1,\y2) -- (E01);
  \NULL{E01}

  \draw[*->] (V1) -- (E12);
  \NULL{E12}

  \draw[*->] (V2) -- (E20);
  \draw[*->] let \p1 = (E20.two), \p2 = (E20.center) in (\x1,\y2) -- (E23);
  \NULL{E23}

  \EMPTYKEY{V3}

\end{tikzpicture}
\end{column}
\end{columns}

\end{frame}
