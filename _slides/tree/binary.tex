\title{Árvores Binárias}

\frame{\maketitle}

\begin{frame}{Árvores binárias}

Uma *árvore binária T* é um conjunto finito de elementos
denominados nós ou vértices, tal que:

- $T\in\emptyset$ e a árvore é dita vazia;
- Existe um nós especial $r$, chamado {\it raiz\/} de $T$, e os
restantes podem ser divididos em dois subconjuntos disjuntos, $T^E_r$
e $T^D_r$, a {\it subárvore esquerda\/} e a {\it direita} de $r$, as
quais são também árvores binárias.

\end{frame}

\begin{frame}{Árvore estritamente binária}

\begin{columns}
\begin{column}{.5\textwidth}
\footnotesize
Uma árvore {\bf estritamente binária} é
uma árvore binária em que cada nó possui
$0$ ou $2$ filhos.
\end{column}

\begin{column}{.5\textwidth}
    \begin{tikzpicture}
\tikzset{T/.style={circle,draw},
sibling angle=60}

\def\shift{1cm}

\node[T] (root) {} []
      child {node[T] [xshift=-.5*\shift] {}}
      child {node[T] [xshift=.5*\shift] {}
             child {node[T] [xshift=-.25*\shift] {}
                    child {node[T][xshift=-.25*\shift] {}}
                    child {node[T][xshift=-.25*\shift] {}}}
             child {node[T] [xshift=.25*\shift] {}}
};

\end{tikzpicture}
                     
\end{column}

\end{columns}

\end{frame}

\begin{frame}{Árvore binária completa}

\begin{columns}
\begin{column}{.5\textwidth}
\scriptsize
Uma {\bf árvore binária completa} apresenta
as seguintes propriedades:
se $v$ é um nó tal que alguma subárvore de $v$ \ é
vazia, então $v$ localiza-se no último (maior) \ ou penúltimo nível da
árvore.
\end{column}

\begin{column}{.5\textwidth}
   \begin{tikzpicture}
\tikzset{T/.style={circle,draw},
sibling angle=60}

\def\shift{1cm}

\node[T] (root) {} []
      child {node[T] [xshift=-.5*\shift] {}
                  child {node[T] [xshift=-.25*\shift] {}}
                  child {node[T] [xshift=-.25*\shift] {}}}
      child {node[T] [xshift=.5*\shift] {}
             child {node[T] [xshift=-.25*\shift] {}
                    child {node[T][xshift=-.25*\shift] {}}
                    child {node[T][xshift=-.25*\shift] {}}}
             child {node[T] [xshift=.25*\shift] {}}
};


\end{tikzpicture}

\end{column}
\end{columns}

\end{frame}

\begin{frame}{Árvore binária cheia}

\scriptsize Um {\bf árvore binária cheia} apresenta as seguintes
propriedades: 
se $v$ é um nó tal que alguma subárvore de $v$ \
é vazia, então $v$ localiza-se no último \
 nível da árvore.

\begin{center}
        \begin{tikzpicture}
\tikzset{T/.style={circle,draw},
sibling angle=60}

\def\shift{3cm}

\node[T] (root) {} []
      child {node[T] [xshift=-.5*\shift] {}
                      child {node[T] [xshift=-.1*\shift] {}
                                     child {node[T] [] {}}
                                     child {node[T] [] {}}
                     }
                     child {node[T] [xshift=.1*\shift] {}
                                    child {node[T] [] {}}
                                    child {node[T] [] {}}
                     }}
      child {node[T] [xshift=.5*\shift] {}
                     child {node[T] [xshift=-.1*\shift] {}
                                    child {node[T][] {}}
                                    child {node[T][] {}}
                     }
                     child {node[T] [xshift=.1*\shift] {}
                                    child {node[T][] {}}
                                    child {node[T][] {}}
                     }
};


\end{tikzpicture}

   \end{center}

\end{frame}

\begin{frame}{Árvore binária: propriedades}

\begin{center}
        \begin{tikzpicture}
\tikzset{
        every node/.style={font=\scriptsize},
        every path/.style={draw},
        T/.style={circle,draw},
        sibling angle=60,
        L/.style={dashed} % level
}

\def\shift{3cm}

\foreach \Y/\L in {2/{$n$ -- n$^{o.}$ de nós},1.5/{$l$ -- nível},1/{$h$ -- altura}} {
         \node at (4,\Y) {\L};
}

\node at (0,2) {$n_l= 2^l = 1+ \sum_{i=0}^{l-1}2^i$};
\node at (0,1) {$n_h= 2^{h+1}-1$};

\node[T] at (0,0) (root) {} []
      child {node[T] [xshift=-.5*\shift] {}
                      child {node[T] [xshift=-.1*\shift] {}
                                     child {node[T] [] {}}
                                     child {node[T] [] {}}
                     }
                     child {node[T] [xshift=.1*\shift] {}
                                    child {node[T] [] {}}
                                    child {node[T] [] {}}
                     }}
      child {node[T] [xshift=.5*\shift] {}
                     child {node[T] [xshift=-.1*\shift] {}
                                    child {node[T][] {}}
                                    child {node[T][] {}}
                     }
                     child {node[T] [xshift=.1*\shift] {}
                                    child {node[T][] {}}
                                    child {node[T][] {}}
                     }
};

\foreach \j/\n/\y in {0/1/.2, 1/2/1.75, 2/4/3.25, 3/8/4.7} {
         \path[L] (-1.75*\shift, -\y) -- (1.75*\shift, -\y)
         node[above] {n\'ivel \j: $2^\j=\n$};
}
\end{tikzpicture}

   \end{center}

 \end{frame}
 
\begin{frame}{Definição do tipo abstrato de dados}

\begin{columns}
\begin{column}{.65\textwidth}
  \lstinputlisting[firstline=4,lastline=14]{src/tree.h}
\end{column}
\begin{column}{.35\textwidth}

\begin{tikzpicture}
\tikzset{every node/.style={minimum width=2.15cm, draw}}

\node[white, fill=black] (K) {chave};
\node[rectangle split,rectangle split horizontal, rectangle split
parts=2, draw] [below of=K,yshift=.45cm] {\nodepart{one} \tt\color{green!40!black} *esq \nodepart{two} \tt\color{blue} *dir};
\end{tikzpicture}

\end{column}
\end{columns}

\end{frame}

\begin{frame}{Percursos em Árvores Binárias: pré-ordem}
       %animategraphics[step]{1}{img/Tpre}{}{}
\begin{center}
 \begin{tikzpicture}
  \input{/tmp/pre}
\end{tikzpicture}
\end{center}

\end{frame}
     
\begin{frame}{Percurso em pré-ordem}
\lstinputlisting[firstline=5,lastline=8]{src/tree.c}
\lstinputlisting[mathescape,firstline=9,lastline=18]{src/tree.c}
\end{frame}

\begin{frame}{Percurso em ordem simétrica}
 %animategraphics[step]{1}{img/Tin}{}{}

\begin{center}
 \begin{tikzpicture}
  % \directlua{treepost()}
  \input{/tmp/in}
\end{tikzpicture}
\end{center}

\end{frame}

\begin{frame}{Percurso em ordem simétrica ({\it in-order\/})}

\lstinputlisting[firstline=5,lastline=8]{src/tree.c}
 \lstinputlisting[firstline=18,lastline=28]{src/tree.c}

\end{frame}

\begin{frame}{Percurso em pós-ordem}

%animategraphics[step]{1}{img/Tpos}{}{}
\begin{center}
  \begin{tikzpicture}
    \input{/tmp/post}
  \end{tikzpicture}
\end{center}

\end{frame}

\begin{frame}{Percurso em pós-ordem}

\lstinputlisting[firstline=5,lastline=8]{src/tree.c}
 \lstinputlisting[firstline=30,lastline=38]{src/tree.c}

\end{frame}

\begin{frame}{Árvore binária de busca}
  
  \begin{tikzpicture}[scale=1,no/.style={circle,draw}]
    \node (n1) at (0,2) [no] {1};
    \node (n2) at (1,1) [no] {2};
    \node (n3) at (2,4) [no] {3};
    \node (n4) at (3,0) [no] {4};
    \node (n5) at (4,1) [no] {5};
    \node (n6) at (5,0) [no] {6};
    \node (n7) at (6,2) [no] {7};
    \draw (n3) -- (n1) -- (n2);
    \draw (n3) -- (n7) -- (n5) -- (n4);
    \draw (n5) -- (n6);
    
    \def\start{-1}
    \def\ending{8}
    \newcounter{level}
    \setcounter{level}{4}
    \foreach \i in {0,...,4}{
      \draw[dotted] (\start,\i-1) -- (\ending,\i-1);
      \draw (\start,\i-0.8) node{\tiny $h=\i$};
      \draw (\ending,\i-0.8) node{\tiny $\ell=\arabic{level}$};
      \addtocounter{level}{-1}
    }
    \draw (\start,5) node{\scriptsize altura};
    \draw (\ending,5) node{\scriptsize nível};
    
    \draw[draw=gray,->,>=latex] (\start-0.5,-1) -- (\ending+0.5,-1);

    \foreach \i in {1,...,7} {
      \draw[draw=gray,dotted] (n\i) -- (\i-1,-1) node[below]{\tiny \i};
    }
  \end{tikzpicture}

\end{frame}

\begin{frame}{Busca: implementação em C}
  
 \lstinputlisting[firstline=58,lastline=67]{src/tree.c}

\end{frame}

\begin{frame}{Leitura Adicional}

\begin{thebibliography}{4}
\bibitem{szwarcfiter}[Szwarcfiter, 1994]
Jayme Luiz Szwarcfiter e Lilian Markezon
\newblock Estrutura de Dados e seus Algoritmos
\newblock Editora LTC, 1994.

\bibitem{taocp3}[Knuth3, 1998]
Donald Erwin Knuth
\newblock The Art of Computer Programming, v. 3, 2$^a$ ed.
\newblock Addison-Wesley, 1998.
\end{thebibliography}

\end{frame}
