% Árvores e Àrvores Binárias
% Adriano J. Holanda
%

Árvores
=======

- Representação hieráquica;
- Generalização de listas lineares.

Definições
=========

Uma \alert{árvore enraizada $T$}, ou simplesmente \alert{árvore}, 
é um conjunto finito de elementos denominados \alert{nós} ou
\alert{vértices} tais que:

- $T=\emptyset$, e a árvore é dita vazia, ou
 existe um nó especial, $r$, chamando \alert{raiz} de $T$; os
restantes constituem um único conjunto vazio ou são divididos em
$m\geq 1$ conjuntos distintos não vazios, as \alert{subárvores} de
$r$, ou simplesmente \alert{subárvores}, cada qual por sua vez uma
árvore.
- Uma \alert{floresta} é um conjunto de árvores~\cite{szwarcfiter}.


Definições
============

- \alert{Nós internos} possuem filhos;
- \alert{Nós externos} ou \alert{folhas} não possuem.


\begin{tikzpicture}
\def\shift{2cm}
\tikzset{
        every node/.style={font=\scriptsize},
        every path/.style={draw},
        T/.style={circle,draw},
        sibling angle=60,
        L/.style={dashed} % level
}

\node[T,white,fill=black] at (0,0) (root) {raiz} []
      child {node[T,fill=brown] [xshift=-.25*\shift] {nó int.}
                      child {node[T,fill=green] [xshift=-.1*\shift] {folha}}}
      child {node[T,fill=green] (intnode2) [xshift=.25*\shift] {folha}
};

\node [right of=intnode2,xshift=\shift] {int. = interno};

\end{tikzpicture}


Representações
=============

\begin{columns}
\begin{column}{.5\textwidth}
hierárquica\\
\bigskip
\begin{tikzpicture}
\tikzset{T/.style={circle,draw},
sibling angle=60}

\def\shift{1cm}

\node[T] (root) {A} []
      child {node[T] [xshift=-.5*\shift] {B}}
      child {node[T] [xshift=.5*\shift] {C}
             child {node[T] [xshift=-.25*\shift] {D}
                             child {node[T] [xshift=-.5*\shift] {G}}
                             child {node[T] [xshift=.5*\shift] {H}}}
                    child {node[T][xshift=-.25*\shift] {E}}
             child {node[T] [xshift=.25*\shift] {F}
                             child {node[T] [xshift=.5*\shift] {I}}}
};

\end{tikzpicture}

\end{column}
\begin{column}{.5\textwidth}
parênteses aninhados\\

\begin{center}
(A(B)(C(D(G)(H))(E)(F(I)))
\end{center}
\end{column}
\end{columns}
