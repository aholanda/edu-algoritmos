% Árvores AVL

Árvores AVL
===========
    
- Solução para a manutenção de árvore de busca equilibrada
    proposta por dois matemáticos russos, G.\ M.\ Adelson-Velsky e E.\
    M.\ Landis, em 1962.

- Árvore binária de busca.
    
- Restrição para a diferença entre as alturas das árvores
    direita e esquerda de cada nó.


Restrição
=========

  $|h_E(n_i)-h_D(n_i)|<1$


Equilíbrio
==========

  \begin{theorem}
    (Adelson-Velsky e Landis). A altura de uma árvore balanceada (AVL) com N nós internos varia
    sempre entre $lg(N+1)$ e $1.4405lg(N+2)-0.3277$.
  \end{theorem}



Inclusão: Caso 1, $h_E(p) > h_D(p)$
==================================

includegraphics{img/06treeapp-avl-single_rotation}

Inclusão: Caso 1.1, $h_E(p) > h_D(p) \land h_E(u) > h_D(u)$
===========================================================

\begin{columns}
\begin{column}{0.5\textwidth}
 includegraphics[scale=0.5]{img/06treeapp-avl-single-rotation-right-before}
\end{column}
\begin{column}{0.5\textwidth}
includegraphics[scale=0.5]{img/06treeapp-avl-single-rotation-right-after}
\end{column}
\end{columns}

Inclusão: Caso 1.2, $h_E(p) > h_D(p) \land h_D(u) > h_E(u)$
===========================================================

\begin{columns}
\begin{column}{0.35\textwidth}
includegraphics[scale=0.375]{img/06treeapp-avl-double-rotation-right-before}
\end{column}
\begin{column}{0.65\textwidth}
$\Rightarrow$fig
includegraphics[scale=0.37]{img/06treeapp-avl-double-rotation-right-after}
\end{column}
\end{columns}

Inclusão: Caso 2, $h_D(p) > h_E(p)$
===================================

includegraphics{img/06treeapp-avl-single_rotation}

Inclusão: Caso 2.1, $h_D(p) > h_E(p) \land h_D(z) > h_E(z)$

\begin{columns}
\begin{column}{0.5\textwidth}
includegraphics[scale=0.5]{img/06treeapp-avl-single-rotation-left-before}
\end{column}
\begin{column}{0.5\textwidth}
includegraphics[scale=0.5]{img/06treeapp-avl-single-rotation-left-after}
\end{column}
\end{columns}

Inclusão: Caso 2.2, $h_D(p) > h_E(p) \land h_E(z) > h_D(u)$
==========================================================

\begin{columns}
\begin{column}{0.35\textwidth}
includegraphics[scale=0.375]{img/06treeapp-avl-double-rotation-left-before}
\end{column}
\begin{column}{0.65\textwidth}
$\Rightarrow$ includegraphics[scale=0.37]{img/06treeapp-avl-double-rotation-left-after}
\end{column}
\end{columns}

Algoritmo para o caso 1
=======================

\small
\begin{block}{Caso 1}
\begin{lstlisting}[basicstyle=\color<2>{gray}]]
procedimento caso1(pt,h)
  ptu := pt^.dir
  se ptu^.bal = 1 then
         pt^.dir := ptu^.esq;   ptu^.esq := pt
         pt^.bal := 0;          pt := ptu
\end{lstlisting}
\pause
\begin{lstlisting}[firstnumber=6,basicstyle=\color<1>{gray}\color<2>{black}]
  else ptv := ptu^.esq
         ptu^.esq := ptv^.dir;  ptv^.dir := ptu
         pt^.dir := ptv^.esq;   ptv^.esq := pt
         se ptv^.bal = 1 then pt^.bal := -1 
         else pt^.bal := 0
         se ptv^.bal = -1 then ptu^.bal := 1 
         else ptu^.bal := 0
         pt := ptv
  pt^.bal := 0; h := F   
\end{lstlisting}
\end{block}


Algoritmo para o caso 2
=======================

\small
\begin{block}{Caso 2}
\begin{lstlisting}[basicstyle=\color<1>{black}\color<2>{gray}]
procedimento caso2(pt,h)
  ptu := pt^.esq
  se ptu^.bal = -1 then
         pt^.esq := ptu^.dir;   ptu^.dir := pt
         pt^.bal := 0;          pt := ptu
\end{lstlisting}
\pause
\begin{lstlisting}[basicstyle=\color{black},firstnumber=6]
  else ptv := ptu^.dir
         ptu^.dir := ptv^.esq;  ptv^.esq := ptu
         pt^.esq := ptv^.dir;   ptv^.dir := pt
         se ptv^.bal = -1 then pt^.bal := 1 
         else pt^.bal := 0
         se ptv^.bal = 1 then ptu^.bal := -1 
         else ptu^.bal := 0
         pt := ptv
  pt^.bal := 0; h := F     
\end{lstlisting}

\end{block}

Algoritmo de inclusão
======================

\scriptsize
\begin{block}{Inclusão}
\onslide<1>{\lstset{basicstyle=\color{black}}}
\onslide<2>{\lstset{basicstyle=\color{gray}}}
\onslide<3>{\lstset{basicstyle=\color{gray}}}
\begin{lstlisting}[basicstyle=\color<1>{black}\color<2>{gray}\color<3>{gray}]
procedimento ins-AVL(x,pt,h)
  se pt = null
     inicio-no(pt)
     h := V
\end{lstlisting}
\pause
\begin{lstlisting}[firstnumber=5,basicstyle=\color<2>{black}\color<3>{gray}]
  else se x = pt^.chave then pare
       se x < pt^.chave then
          ins-AVL(x,pt^.esq,h)
          se h then
             caso pt^.bal seja
             1:   pt^.bal := 0
             0:   pt^.bal := -1;   h := F
             -1:  caso1(pt,h)     % rebalanceamento
\end{lstlisting}
\pause
\begin{lstlisting}[firstnumber=13,basicstyle=\color{black}]
             else ins-AVL(x,pt^.dir,h)
            se h then
            caso pt^.bal seja
            -1:  pt^.bal := 0;  h := F
            0:   pt^.bal := 1
            1:   caso2(pt,h)    % rebalanceamento             
\end{lstlisting}
\end{block}
