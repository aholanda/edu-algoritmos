\lstset{basicstyle=\scriptsize,frame=single}

\exercise Utilize o código {\tt fila\_vetor.c}
(Listagem~\ref{lst:FilaVetorC}) e faça uma tabela contendo o valor
das variáveis {\tt p->ini}, {\tt p->n} e dos elementos do vetor{\tt
  p->v[TAM\_VETOR]} em cada passo da execução do código a seguir:

\begin{center}
  \begin{lstlisting}
    int main() {
      Fila *p = fila_cria();
      fila_insere(p, 5);
      fila_insere(p, 11);
      fila_insere(p, 8);
      fila_retira(p);
      fila_insere(p, 9);
      fila_insere(p, 4);
      fila_retira(p);
      fila_insere(p, 101);
      fila_retira(p);
      fila_retira(p);
      fila_insere(p, 22);
      fila_retira(p);
      fila_retira(p);
      fila_libera(p);
    }
  \end{lstlisting}
\end{center}

\exercise Execute a função {\tt main()} do exercício anterior usando o
código {\tt fila\_lista.c} (Listagem~\ref{lst:FilaListaC}) e faça um
desenho dos ponteiros {\tt p->ini} e {\tt p->fim}, bem como das listas
contendo as chaves após cada execução de cada instrução de {\tt main()}.

\exercise~Qual a ordem de complexidade (desempenho) em {\tt
  fila\_vetor.c} e {\tt fila\_lista.c} das funções:

\begin{enumerate}[a)]
\item {\tt fila\_cria()}
\item {\tt fila\_insere()}
\item {\tt fila\_retira()}
\item {\tt fila\_libera()}
\end{enumerate}

\exercise Faça um programa que simule um fila de atendimento com
senhas de 1 a 100. O programa deve exibir um menu ao controlador com
as seguinte opções:

\begin{enumerate}
\item Atribuir senha ao cliente que chega;
\item Atender o próximo da fila;
\item Sair do programa.
\end{enumerate}

\noindent Na opção 1, deverá ser atribuída uma senha de 1 a 100,
respeitando uma sequência de contagem, ao cliente e colocado na
fila. Na opção~2, o próximo senha a ser atendida deverá ser retirada
da fila.

\pagebreak\section*{Anexo}

\lstinputlisting[caption={Listagem do arquivo de cabeçalho para o tipo de dados fila.},label={lst:FilaH}]{fila.h}
\lstinputlisting[caption={Listagem do código que implementa o tipo de dados fila usando vetor.},label={lst:FilaVetorC}]{fila_vetor.c}
\lstinputlisting[caption={Listagem do código que implementa o tipo de dados fila usando lista encadeada},label={lst:FilaListaC}]{fila_lista.c}