\documentclass[10pt]{../notes}

%\newgeometry{left=1.75cm,right=1.75cm,top=1.75cm,bottom=1.75cm}

\title{Algoritmo I}
\author{prof. Adriano J. Holanda}
\date{13 de maio de 2016}

\def\citafarrer{{\tt [Farrer {\it et al.}]}}

\lstset{
  emph={Algoritmo,declare,então,falso,Algoritmo,fim,interrompa,lógico,numérico,repita,se,senão,verdadeiro},emphstyle=\underbar,
  literate={:=}{{$\gets$\ }}1 {<=}{{$\leq$\ }}1 {>=}{{$\geq$\ }}1 {<>}{{$\neq$\ }}1 {/and}{{$\land$\ }}1 {/or}{{$\lor$\ }}1 {\#}{{$\ne$\ }}1 {/not}{{$\lor$\ }}1 
}

\begin{document}
\small

\maketitle

\section*{Lista de exercícios 4}

\exercise Escreva um algoritmo que armazene o triângulo de Pascal em
uma matriz $N\times N$, onde N é fornecido pelo usuário. Por exemplo, 
para $N=7$, o triângulo de Pascal gerado seria

\begin{center}
\begin{tabular}[h]{c|ccccccc}
  & \bf 0 & \bf 1 & \bf 2 & \bf 3 &\bf 4 &\bf 5&\bf 6\\\hline
\bf 0 & 1&  &   &   &  &  & \\
\bf 1 & 1& 1&   &   &  &  & \\
\bf 2 & 1& 2&  1&   &  &  & \\
\bf 3&  1& 3&  3&  1&  &  & \\
\bf 4& 1&  4&  6&  4&  1&  & \\
\bf 5& 1&  5& 10& 10&  5&  1& \\
\bf 6& 1&  6& 15& 20&  15&  6& 1\\
\end{tabular}
\end{center}

\exercise Faça um algoritmo que verifique se uma matriz $N\times N$
fornecida pelo usuário é uma {\bf matriz mágica}. Uma matriz mágica é
uma matriz em que a soma de cada coluna, de cada linha e das duas
diagonais são iguais. Por exemplo, a seguinte matriz {\tt M[3][3]} é
uma matriz mágica:

\begin{center}
  $$M = \pmatrix{
    2 & 7 & 6 \cr
    9 & 5 & 1 \cr
    4 & 3 & 8 \cr
}. $$
\end{center}

\exercise Escreva um algoritmo que dada uma matriz $A\rightarrow N\times M$ 
({\tt A[N][M]}), fornecida pelo usuário, gere sua transposta 
$T \rightarrow M\times N$ ({T[M][N]}). Por exemplo, a transposta da matriz 

\begin{center}
 $$A = \pmatrix{   
   1 & 3\cr
    2 & 5 \cr
    8 & -1 \cr
}$$
\end{center}

\noindent é a matriz

\begin{center}
$$T = \pmatrix{
    1 & 2 & 8 \cr
    3 & 5 & -1 \cr
}. $$
\end{center}

\exercise Uma matriz $A\rightarrow N\times N$ é de permutação se em cada 
linha e em cada coluna houver $N-1$ elementos iguais a zero e apenas um 
único elemento igual a $1$. Por exemplo, a matriz 

\begin{center}
  $$A = \pmatrix{
    0 & 1 & 0 \cr
    1 & 0 & 0 \cr
    0 & 0 & 1 \cr
  }$$
\end{center}

\noindent é de permutação. Faça um algoritmo que fornecida uma matriz $N\times N$ 
pelo usuário, verifique se ela é de permutação.

\subsection*{Solução}

\noindent {\bf Triângulo de Pascal} 

\begin{lstlisting}
  Algoritmo
  declare PAS[N][N], N, I, J numérico
  I := 0
  J := 0
  LEIA(N)
  repita
        se I >= N
            então interrompa
        fim se
        repita
            se J >= N
                então interrompa
            fim se
            se I >= J
                se J == 0
                    então PAS[I][J] := 1
                    senão PAS[I][J] := PAS[I-1][J] + PAS[I-1][J-1]
                fim se
            fim se
            J := J + 1
        fim repita        
        I := I + 1
  fim repita

  fim Algoritmo
\end{lstlisting}

\pagebreak
\noindent {\bf Matriz mágica} 

\begin{lstlisting}
  Algoritmo
  declare MAT[N][N], SOM[N+N+2], N, I, J numérico
          MAGICA                         lógico
  MAG := VERDADEIRO 
  I := 0
  J := 0
  LEIA(N)
  LEIA(MAT[0][0],...,MAT[N-1][N-1])
  repita
        se I >= N
            então interrompa
        fim se
        repita
            se J >= N
                então interrompa
            fim se
            SOM[I] := MAT[I][J] {soma da linha}
            SOM[N+J] := MAT[I][J] {soma da coluna}
            se I == J
                    então SOM[N+N] := MAT[I][J] {soma da diagonal principal}
            fim se
            se I == J
                    então SOM[N+N+1] := MAT[I][J] {soma da diagonal secundaria}
            fim se
            J := J + 1
        fim repita        
        I := I + 1
  fim repita

  I := 1
  repita
        se I >= N
            então interrompa
        fim se
        se SOM[I] # SOM[I-1]
            então MAGICA = FALSO
                  interrompa
        fim se
  fim repita

  se MAGICA = VERDADEIRO
      entao ESCREVE('A matriz é magica')
      senão ESCREVE('A matriz não é magica')
  fim se 
fim Algoritmo
\end{lstlisting}

\pagebreak
\noindent {\bf Matriz tranposta} 

\begin{lstlisting}
  Algoritmo
  declare A[N][M], T[M][N], N, M, I, J numérico
  I := 0
  J := 0
  LEIA(N)
  LEIA(A[0][0],...,A[N-1][M-1])
  repita
        se I >= M
            então interrompa
        fim se
        repita
            se J >= N
                então interrompa
            fim se
            T[I][J] := A[J][I] 
            J := J + 1
        fim repita        
        I := I + 1
  fim repita
  
  ESCREVA(T[0][0],...,T[M-1][N-1])
fim Algoritmo
\end{lstlisting}

\pagebreak
\noindent {\bf Matriz de permutação} 

\begin{lstlisting}
  Algoritmo
  declare MAT[N][N], SOM[N+N], N, I, J numérico
          PERM                           lógico
  PERM := VERDADEIRO 
  I := 0
  J := 0
  LEIA(N)
  LEIA(MAT[0][0],...,MAT[N-1][N-1])
  repita
        se I >= N
            então interrompa
        fim se
        repita
            se J >= N
                então interrompa
            fim se
            SOM[I] := MAT[I][J] {soma da linha}
            SOM[N+J] := MAT[I][J] {soma da coluna}

            J := J + 1
        fim repita        
        I := I + 1
  fim repita

  I := 1
  repita
        se I >= N+N
            então interrompa
        fim se
        se SOM[I] # 1
            então PERM = FALSO
                  interrompa
        fim se
  fim repita

  se MAG = VERDADEIRO
      entao ESCREVE('A matriz é de permutação')
      senão ESCREVE('A matriz não é permutação')
  fim se 
fim Algoritmo
\end{lstlisting}


\end{document}

\section*{Lista de exercícios 3}

\exercise Escreva um algoritmo que leia $20$ valores numéricos do
usuário em um vetor chamado {\tt VET[20]} e imprima-os em ordem
inversa.

\exercise Escreva uma algoritmo que leia $132$ valores de salário da
entrada-padrão (usuário) em um vetor chamado {\tt SAL[132]} e imprima
o maior, o menor e a média de salário.

\exercise Faça um algoritmo que leia 32 valores de 2 vetores de
componentes {\tt A[32]} e {\tt B[32]} e calcule o produto escalar
$A. B$. O produto escalar é calculado usando a fórmula

$$ A.B = \sum_{i=1}^n a_ib_i = (a_1b_1 + a_2b_2 + \ldots + a_n b_n).$$

\exercise Faça um algoritmo que gere uma tabela com $N$ termos da
sequência de Fibonacci em um vetor chamado {\tt FIBO[N]}. O valor de
$N$ deve ser entrado pelo usuário. Cada termo da sequência de
Fibonacci é a soma dos dois anteriores conforme mostrado a seguir:

$$1, 1, 2, 3, 5, 8, 13, 21, 34, 55, \ldots$$

\exercise Faça uma algoritmo que leia $N$ valores de despesas e 
atribua-as a um vetor chamado {\tt DESPESAS[N]}, leia $M$ valores 
de receitas em um vetor chamado {\tt RECEITAS[M]} e escreva:

\begin{enumerate}[a)]
\item Valor total das despesas;
\item Valor total das receitas;
\item Saldo.
\end{enumerate}

\end{document}
%{\footnotesize\noindent Entrega: {\bf 6/4/2016}.}

\paragraph{Exercício 1.} \citafarrer~Fazer um algoritmo que calcule e escreva o valor de $S$ onde:

$$ S = {1\over 1} - {2\over 4} + {3\over 9} - {4\over 16} + {5\over 25} 
- {6\over 36} + \ldots -{10\over 100}.$$

\paragraph{Exercício 2.} \citafarrer~Fazer um algoritmo que calcule e escreva a soma dos 50 
primeiros termos da série:

$$ {1000\over 1} - {997\over 2} + {994\over 3} - {991\over 4} + \ldots$$

\paragraph{Exercício 3.} \citafarrer~Fazer um algoritmo que
\begin{enumerate}
\item Leia o valor de $x$ da entrada-padrão;
\item Calcule e escreva o valor do seguinte somatório:
\end{enumerate}

$$ {x^{25}\over 1} - {x^{24}\over 2} + {x^{23}\over 3} - {x^{22}\over 4} + \ldots + {x\over 25}.$$

\paragraph{Exercício 4.} \citafarrer~Fazer um algoritmo que calcule o
volume de uma esfera em função do raio $R$. O raio deverá variar de 0
a 20~cm de 0,5 em 0,5~cm,

$$ V = {4\over 3}\pi R^3.$$

\paragraph{Exercício 5.} \citafarrer~Fazer um algoritmo que leia a
altura e sexo (``MASCULINO'' ou ``FEMININO'') de 50 pessoas. Faça 
um algoritmo que calcule e escreva:
\begin{enumerate}[a)]
\item A maior e a menor altura do grupo;
\item A média de altura das mulheres;
\item O número de homens.
\end{enumerate}

\paragraph{Exercício 6.} \citafarrer~Uma empresa decidiu fazer um
levantamento em relação aos candidatos que se apresentarem para
preenchimento de vagas no seu quadro de funcionários, utilizando
processamento eletrônico. Supondo que você seja o desenvolvedor
encarregado desse levantamento, fazer um algoritmo que:

\begin{itemize}
\item Leia um conjunto de dados para $100$ candidatos contendo:
  \begin{enumerate}[a)]
  \item Número de inscrição do candidato;
  \item Idade;
  \item Sexo (``MASCULINO'' ou ``FEMININO'');
  \item Experiência no serviço (``SIM'' ou ``NÃO'').
  \end{enumerate}
\item Calcule e escreva:
  \begin{enumerate}[a)]
  \item O número de candidatos do sexo feminino;
  \item O número de candidatos do sexo masculino;
  \item Idade média dos homens que já têm experiência no serviço;
  \item Porcentagem de homens com mais de 45 anos entre o total de homens;
  \item Número de mulheres que têm idade inferior a 35 anos e com experiência 
    no serviço;
  \item A menor idade entre mulheres que já têm experiência no serviço.
  \end{enumerate}
\end{itemize}

\end{document}

\section*{Lista de exercícios 1}

\paragraph{Exercício 1.} [Farrer {\it et al.}]~Assinale com um X os
identificadores válidos:

\begin{tabular}[h]{lll}
  (\ ) VALOR & (\ ) SALÁRIO-LÍQUIDO & (\ ) B248 \\
  (\ ) X2 & (\ ) SALÁRIO-LÍQUIDO & (\ ) A1B2C3 \\
  (\ ) XYZ & (\ ) NOTA*ALUNO & (\ ) M[A] \\
  (\ ) ``MEDIA'' & (\ ) AHH! & (\ ) N{A} \\
  (\ ) SALA12 & (\ ) 3A & (\ ) KM/H \\
\end{tabular}

\paragraph{Exercício 2.} Faça a tabela-verdade para as expressões lógicas:

\begin{enumerate}[a)]
\item $P \land Q \land R$
\item $P \land Q \lor R$
\item $P \lor Q \land R$
\item $P \lor Q \lor R$
\item $P \lor Q \lor \neg R$
\end{enumerate}

\paragraph{Exercício 3.} Faça um algoritmo que leia três valores e escreva-os 
em ordem decrescente.

\paragraph{Exercício 4.} Faça um algoritmo que leia três notas de um
aluno e escreva se ele foi aprovado ou reprovado. Para ser aprovado a
média das notas deve ser maior ou igual a cinco.

\paragraph{Exercício 5.} Faça um algoritmo que leia três valores e escreva o maior
valor.

\paragraph{Exercício 6.} [{Farrer\it et al.}] Após a execução do trecho a seguir de um algoritmo,

\begin{lstlisting}
  Algoritmo
  .
  .
  se A2 <= B3
  então T := verdadeiro
  senão T := falso
  fim se
  .   C := T
  .
  fim Algoritmo
\end{lstlisting}

\noindent em C estará armazenado o valor false se:

\begin{enumerate}[a)]
\item A2 $<$ B3
\item A2 $\leq$ B3
\item A2 $\geq$ B3
\item A2 $>$ B3
\item A2 $=$ B3
\end{enumerate}


\paragraph{Exercício 7.} Dado o algoritmo a seguir:

\begin{lstlisting}
  Algoritmo
  declare A,B,C,I,J,K numérico
  A := 36
  C := 2
  I := 5
  B := SQRT(A)
  J := C * 3/4
  se B > J
  então K := 8 * I / (POW(6, 2)/C)
  senão K := A + I/A - I
  fim se
  ESCREVA(B, J, K)
  fim Algoritmo
\end{lstlisting}

\noindent quais valores serão escritos?

\paragraph{Exercício 8.} [Farrer {\it et al.}] Dado o algoritmo a seguir:

\begin{lstlisting}
  Algoritmo
  declare A, B, I, M numérico
  LEIA(M)
  se M <> 0
  então I := TRUNCA(M/12)
  A := M/12
  B := ARREDONDA(M/12)
  se RESTO(M, 12) >= 6
  então I := I + 1
  fim se
  ESCREVA(A, B, I)
  fim se
  fim Algoritmo
\end{lstlisting}

\noindent Que valores seriam escritos se, em sucessivas execuções, fossem lidos 
os valores 30, 19, 27, 60, 0?

\paragraph{Exercício 9.} [Farrer {\it et al.}] Quais os valores escritos pelo algoritmo a seguir:

\begin{lstlisting}
  Algoritmo
  declare A,B,C,MAT,X,XX,Y,YY numérico
  X := 0
  XX := 0
  Y := 0
  YY := 0
  A := 2
  B := 5
  C := 4
  MAT := {o último algarismo do dia em que você nasceu}
  se MAT >= 5
  então X := SQRT(A + B) + C/2 * A
  Y := RESTO(B, A)/A - QUOCIENTE(C, POW(B, 2))
  senão XX := QUOCIENTE(B, A)/POW(A, 2) + RESTO(POW(B, 2), C)
  YY := A/2 * C + POW(3 + A, 2)/A
  fim se
  ESCREVA(X, Y, XX, YY)
  fim Algoritmo
\end{lstlisting}

\end{document}