
\noindent Lista a ser entregue no dia {\bf 12 de abril de 2016} antes da prova.

\section*{Busca sequencial}


\exercise~Modifique a função {\tt BuscaSequencial()}, trocando o laço
{\tt for} pelo laço {\tt while}:

\begin{lstlisting}
  int BuscaSequencial(int k, int N, int v[])
  {
    int i;
    for (i = 0; i < N; i++) { 
      if (v[i] == k) 
      return i;       /* SUCESSO */  
    }                                
    return -1;              /* FALHA */
  }  
\end{lstlisting}


\section*{Busca binária}

\exercise Faça um programa que execute o algoritmo da busca binária
recursivamente. Qual a complexidade do algoritmo de busca binária
recursiva? Justifique sua resposta.

\exercise~Faça uma tabela de valores $\lfloor log_2n \rfloor$ para
$n=10, 10^2, 10^3, 10^4, 10^5$.

\exercise~Se t unidades de tempo são necessárias para fazer uma busca
binária em um vetor com $n$ elementos, quantas unidades de tempo são
necessárias para fazer uma busca em $n^2$ segundos.

\exercise~[Feofiloff] {\sc Overflow Aritmético.}~Se o número de
elementos do vetor $v[0..n-1]$ estiver próximo de {\tt INT\_MAX}, o
código da função {\tt BuscaBinaria} pode não funcionar corretamente
ao calcular o valor da expressão $(e+d)/2$, em virtude de um {\it
  overflow\/} aritmético. Como evitar esta possibilidade de
inconsistência?

\begin{lstlisting}
  int BuscaBinaria(int k, int N, int v[])
  {
    int e, m, d;            /* esquerda, meio, direita */
    e = -1;
    d = N;
    while (e <= d) {
      m = (e + d) / 2;
      if (k > v[m]) e = m + 1;
      else if (k < v[m]) d = m - 1;
      else return m;
    }
    return -1;
  }
\end{lstlisting}

\pagebreak
\exercise~[Feofiloff] Mostre que a seguinte variante de função {\tt
  busca\_binariaB} funciona corretamente.

\begin{lstlisting}
  e = 0; d = n;
  while (e < d) { /* $v[e-1] < k \leq v[d]$ */
    m = (e + d)/2;
    if (v[m] < k) e = m + 1;
    else d = m;
  } /* $e==d$ */
  return d;
\end{lstlisting}

\noindent O que acontece se trocarmos ``{\tt while(e < d)}'' por
``{\tt while(e <= d)}''? E se trocarmos ``{\tt (e+d)/2}'' por
``{\tt (e-1+d)/2}''.

\exercise~[Feofiloff] Mostre que a seguinte versão da função {\tt
  busca\_binariaB} funciona corretamente. Ela é um pouco menos
elegante que as anteriores.

\begin{lstlisting}
  e = 0; d = n - 1;
  while (e <= d) { /* $v[e-1] < k \leq v[d+1]$ */
    m = (e + d)/2;
    if (v[m] < k) e = m + 1;
    else d = m - 1;
  } /* $e==d+1$ */
  return d+1;
\end{lstlisting}

\exercise~[Feofiloff] A seguinte alternativa para a função {\tt
  busca\_binariaB} funciona corretamente? O que acontece se trocarmos
``{\tt (e+d)/2}'' por ``(e+d+1)/2''?

\begin{lstlisting}
  e = -1; d = n - 1;
  while (e < d) { 
    m = (e + d)/2;
    if (v[m] < k) e = m;
    else d = m - 1;
  }
  return d+1;
\end{lstlisting}

\end{document}

\exercise~Troque a estrutura de dados utilizada para armazenar as
chaves. Ao invés de utilizar vetor, utilize uma lista encadeada da
forma

\medskip
\begin{lstlisting}
  struct no {
    /* identificador usado na  comparacao */
    int chave;
    /* aponta para o proximo elemento da lista */
    struct no *prox;
  };
\end{lstlisting}
\bigskip

\noindent e implemente as funções:

\begin{enumerate}
\item {\tt busca\_sequencialS()};
\item {\tt busca\_sequencialQ()};
\item {\tt busca\_sequencialT()}.
\end{enumerate}

\noindent Analise as vantagens e desvantagens da utilização de cada
estrutura (vetor e lista encadeada), bem como seu desempenho (complexidade).


\exercise~A busca binária resolve o problema da busca em um vetor
ordenado, mantendo a faixa em que a chave $k$ pode estar dentro do
vetor. Inicialmente, a faixa é todo vetor, que vai sendo reduzida pela
comparação da chave com o elemento do meio, e pelo descarte da metade
em que a chave não se encontra. O processo continua até que a chave
seja encontrada no vetor, ou até que a faixa esteja vazia, indicando
que a chave não se encontra. Em uma tabela com $N$ elementos, a busca
pode levar até $log_2N$ comparações. 
Converta a descrição da busca binária em um programa usando qualquer
linguagem de programação ou pseudo-código.\par
