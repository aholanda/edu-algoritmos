\section*{Pilhas}

\exercise Desenhe a evolução da {\tt Pilha p} após a execução das
seguintes chamadas de função: {\tt pilha\_push(p, 2)}, {\tt
  pilha\_push(p, 11)}, {\tt pilha\_push(p, 64)}, {\tt pilha\_push(p,
  81)}, {\tt pilha\_push(p, 55)}, {\tt pilha\_pop(p)}, {\tt
  pilha\_push(p, 14)}, {\tt pilha\_pop(p)}, {\tt pilha\_pop(p)}, {\tt
  pilha\_pop(p)}, {\tt pilha\_pop(p)}.

\exercise Faça uma função em \CEE{} chamada {\tt pilha\_compara(Pilha
  *p1, Pilha *p2)} que receba duas pilhas {\tt p1} e {\tt p2} como
parâmetros, comparando-as e retornando a porcentagem de elementos que
elas têm em comum na mesma posição.

\exercise Faça uma programa em \CEE\ que receba uma sequência de
caracteres do usuário e inverta-a, usando uma pilha, imprimindo o
resultado na tela.

\exercise Implemente uma função em \CEE\ chamada {\tt
  testa\_palindro(char *s, int n)} que verifique, usando uma pilha, se
uma sequência de caracteres {\tt s} de tamanho {\tt n} é um palíndromo
(simétrica), ou seja, a primeira metade da sequência é igual à segunda
metade invertida. A função deve retornar {\tt 1} se a palavra for um
palíndromo e {\tt 0} caso contrário. Exemplos de palíndromo: arara,
osso, reger, ralar, reviver, salas, somos, Natan, Mussum, mirim,
anilina.

%% Local variables:
%% TeX-master: main
%% End: